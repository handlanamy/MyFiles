%%%%%%%%%%%%%%%%%%%%%%%%%%%%%%%%%%%%
% Template Made by Amy Handlan
% First Draft: September 2020
% Uploaded: September 2021
%%%%%%%%%%%%%%%%%%%%%%%%%%%%%%%%%%%%
\documentclass[aspectratio=1610, english, 12pt, usenames, dvipsnames]{beamer}
\usepackage{xcolor}
\usepackage{tikz}
\usepackage{mathptmx}
\usepackage[T1]{fontenc}
\usepackage[latin9]{inputenc}
%\synctex=-1
\usepackage{amsmath, mathtools}
\usepackage{amssymb}
\usepackage{graphicx}
\usepackage{multirow}
\usepackage{hyperref}
\usepackage{amsmath}
\usepackage{enumerate}
\usepackage{lmodern}
\usepackage{color}
\usepackage{multicol,makecell}
\usepackage{lipsum}
\usepackage{booktabs} %so can use \midrule 
\usepackage{tabularx}
\providecommand{\tabularnewline}{\\}
\usepackage[normalem]{ulem}
\usetikzlibrary{shapes, arrows, matrix,chains,positioning,decorations.pathreplacing, backgrounds, tikzmark}
\DeclarePairedDelimiter\norm{\lVert}{\rVert}%
\usepackage{soul}
\usepackage{centernot}


%%%%%%%%%%%%%%%%%%%%%%%%%%%%%%%%%%%%%%%%%%%%%%%%%%%%%%%%%%%%
%%%%%%%%%% Base Slide Template and Colors %%%%%%%%%%%%%%%%%%
%%%%%%%%%%%%%%%%%%%%%%%%%%%%%%%%%%%%%%%%%%%%%%%%%%%%%%%%%%%%

%%%%%%%%%%%%% Base Theme %%%%%%%%%%%%%%%%%%%%%%%%%%%%%%%%%
\usetheme{Frankfurt}

%%%%%%%%%%%%% Define Colors %%%%%%%%%%%%%%%%%%%%%%%%%%%%%%%%%

% NEURTAL BLUE THEME %%%
\definecolor{basecolor}{RGB}{0,75,150} % Navy Blue
\colorlet{darkbasecolor}{basecolor!60!black} % Section header slightly darker
\colorlet{highlightcolor}{white} % highlightcolor = stripe, make white to erase it

%%% Assign Colors to the Theme%%%
\usecolortheme[named=basecolor]{structure}
\setbeamercolor{section in head/foot}{fg=white, bg=darkbasecolor}

%other colors for quick reference and for highlighting
\newcommand{\red}[1]{\textcolor{red}{#1}}
\newcommand{\blue}[1]{\textcolor{blue}{#1}}
\sethlcolor{yellow} %HIGHLIGHTING
\renewcommand<>{\hl}[1]{\only#2{\beameroriginal{\hl}}{#1}}

%%%%%%%%%%%%% Title Page Style %%%%%%%%%%%%%%%%%%%%%%%%%%%%%%%%%
\setbeamertemplate{title page}[default][colsep=-4bp, shadow=false]

%%%%%%%%%%%%% Frame Title Style %%%%%%%%%%%%%%%%%%%%%%%%%%%%%%%%%
\setbeamertemplate{frametitle}{	\nointerlineskip
	\begin{beamercolorbox}[sep=0.3cm,ht=1.8em,wd=\paperwidth, shadow=False]{frametitle}
		\begin{tikzpicture}[overlay,remember picture]
			\node[anchor=west, align=left] at (-.2,0.17){\insertframetitle};
		\end{tikzpicture}
\end{beamercolorbox}}

%%%%%%%%%%%%% Option to Widen Slide %%%%%%%%%%%%%%%%%%%%%%%%%%%%%%%%%
\newcommand\Wider[2][3em]{%
	\makebox[\linewidth][c]{%
		\begin{minipage}{\dimexpr\textwidth+#1\relax}
			\raggedright#2
\end{minipage}}}

%%%%%%%%%%% Spacing Shortcuts %%%%%%%%%%%%%%
\newcommand{\bs}{\bigskip{}}
\newcommand{\ms}{\medskip{}}

%%%%%%%%%%%%% Header Style (with Sections) %%%%%%%%%%%%%%%%%%%%%%%%%%%%%%%%%
\setbeamertemplate{headline}{ \begin{beamercolorbox}{section in head/foot}
		\vskip2pt\insertsectionnavigationhorizontal{\paperwidth}{}{}\vskip2pt
\end{beamercolorbox}}

%%%%%%%%%%%%% Footer Style %%%%%%%%%%%%%%%%%%%%%%%%%%%%%%%%%
% To remove the navigation symbols from the bottom of slides%
\setbeamertemplate{navigation symbols}{} 
%Custom Footer
\setbeamertemplate{footline}{
	\hbox{%
		\setbeamercolor{footercolor}{bg=white,fg=white!50!black}
		\begin{beamercolorbox}[wd=.333333\paperwidth,ht=2.25ex,dp=2ex,left]{footercolor}%
			\hspace*{2ex} \insertshortauthor\hspace*{.5ex}  (\insertshortinstitute)
		\end{beamercolorbox}%
		\begin{beamercolorbox}[wd=.333333\paperwidth,ht=2.25ex,dp=2ex,center]{footercolor}%
			\insertshorttitle
		\end{beamercolorbox}%
		\begin{beamercolorbox}[wd=.333333\paperwidth,ht=2.25ex,dp=2ex,right]{footercolor}%
			\insertframenumber{}
			%			 / \inserttotalframenumber %uncomment to add total slides
			\hspace*{4ex} 
	\end{beamercolorbox}}%
	\vskip0pt}

%%%%%%%%%%%%% Bullet Style %%%%%%%%%%%%%%%%%%%%%%%%%%%%%%%%%
\setbeamertemplate{enumerate items}[default]
\setbeamertemplate{itemize item}[triangle]
\setbeamertemplate{itemize subitem}[circle]
\setbeamertemplate{section in toc}[circle] %default, circle, square
\usepackage{varwidth}

%%%%%%%%%%%%%%%%% Footnote size %%%%%%%%%%%%%%%%%%%%
\setbeamerfont{footnote}{size=\tiny}
\renewcommand*{\thefootnote}{\fnsymbol{footnote}}


%%%%%%%%%%%%%%%%%%%%%%%%%%%%%%%%%%%%%%%%%%%%%%%%%%%%%%%%%%%%
%%%%%%%%%%%%%%%% Title Page %%%%%%%%%%%%%%%%%%%%%%%%%%%%%%%%
%%%%%%%%%%%%%%%%%%%%%%%%%%%%%%%%%%%%%%%%%%%%%%%%%%%%%%%%%%%%
\newcommand\makebeamertitle{\frame{\maketitle}}%
\title[short title]{\textbf{Title}\\ Subtitle}
%\subtitle{ }
\author[last name]{Author Name\footnote{Email: ,  Website: \url{www.google.com}}}

\institute[Institution]{Institution} 
\date{\today}


%%%%%%%%%%%%%%%%%%%%%%%%%%%%%%%%%%%%%%%%%%%%%%%%%%%%%%%%%%%%
%%%%%%%%%%%%%%%%%%% REFERENCES %%%%%%%%%%%%%%%%%%%%%%%%%%%%%
%%%%%%%%%%%%%%%%%%%%%%%%%%%%%%%%%%%%%%%%%%%%%%%%%%%%%%%%%%%%
\usepackage{appendixnumberbeamer} 
\usepackage[longnamesfirst, sort]{natbib}
\bibliographystyle{elsarticle-harv}
\urlstyle{same} %makes url the same font as text, not courier default
\newcommand{\doi}[1]{\url{#1}}


\makeatletter
\makeatother

%%%%%%%%%%%%%%%%%%%%%%%%%%%%%%%%%%%%%%%%%%%%%%%%%%%%%%%%%%%%
%%%%%%%%%%%%%%% DOCUMENT START %%%%%%%%%%%%%%%%%%%%%%%%%%%%%
%%%%%%%%%%%%%%%%%%%%%%%%%%%%%%%%%%%%%%%%%%%%%%%%%%%%%%%%%%%%
\begin{document}
	
	\section{Introduction}
	
	\begin{frame}
		\maketitle
	\end{frame}
	
	
	
	
	\begin{frame}{Introduction} \label{introduction}
		\begin{itemize}
			\item Motivation: \ms 
			\begin{enumerate}
				\item one \ms 
				\item two \ms 
				\item three \bs 
			\end{enumerate}
			\item This Project:\ms
			\begin{itemize}
				\item Button to Appendix \hyperlink{appendixgraphs}{\beamerbutton{Appendix: Graphs}} \bs
			\end{itemize}
		\end{itemize}
	\end{frame}
	
	
	%%%%%%%%%%%%%%%%%%%%%%%%%%%%%%%%%%%%%%%%%%%%%%%%%%%%%%%%%%%%%%%%%%%%%%%%%%%%%%%%%%%%%%%%%%
	%%%%%%%%%%%%%%%%%%%%%%%%%%%%% TABLE OF CONTENTS AFTER INTRO %%%%%%%%%%%%%%%%%%%%%%%%%%%%%%
	%%%%%%%%%%%%%%%%%%%%%%%%%%%%%%%%%%%%%%%%%%%%%%%%%%%%%%%%%%%%%%%%%%%%%%%%%%%%%%%%%%%%%%%%%%
	%% Comment out this section if you do not want TOC transition slides *********************
	\setbeamertemplate{section in toc}{%
		\leavevmode\leftskip=2ex%
		\llap{%
			\usebeamerfont*{section number projected}%
			\usebeamercolor{section number projected}%
			\begin{pgfpicture}{-1ex}{0ex}{1ex}{2ex} % Section number in circle
				\color{bg}
				\pgfpathcircle{\pgfpoint{0pt}{.75ex}}{1.2ex}
				\pgfusepath{fill}
				\pgftext[base]{\color{fg}\inserttocsectionnumber}
			\end{pgfpicture}\kern1.25ex%
		}%
		\underline{\inserttocsection}\par%
	}
	
	\AtBeginSection[]
	{ 
		\begin{frame}[noframenumbering, plain]{Presentation Outline}
			\tableofcontents[currentsection]
		\end{frame}
	}
	%%%%%%%%%%%%%%%%%%%%%%%%%%%%%%%%%%%%%%%%%%%%%%%%%%%%%%%%%%%%%%%%%%%%%%%%%%%%%%%%%%%%%%%%%%
	%%%%%%%%%%%%%%%%%%%%%%%%%%%%%%%%%%%%%%%%%%%%%%%%%%%%%%%%%%%%%%%%%%%%%%%%%%%%%%%%%%%%%%%%%%
	
	\section{Section A}
	
	\begin{frame}{Slide in Section A}
		
	\end{frame}
	
	
	\section{Section B}
	
	\begin{frame}{Slide in Section B}
		
	\end{frame}
	
	
	
	\section{Conclusion}
	
	\begin{frame}{Conclusion}
		
	\end{frame}
	
	
	
	
	
	\setbeamertemplate{footline}{
		\hbox{%
			\setbeamercolor{footercolor}{bg=white,fg=white!50!black}
			\begin{beamercolorbox}[wd=.333333\paperwidth,ht=2.25ex,dp=2ex,left]{footercolor}%
				\hspace*{2ex} \insertshortauthor\hspace*{.5ex}  (\insertshortinstitute)
			\end{beamercolorbox}%
			\begin{beamercolorbox}[wd=.333333\paperwidth,ht=2.25ex,dp=2ex,center]{footercolor}%
				\insertshorttitle
			\end{beamercolorbox}%
			\begin{beamercolorbox}[wd=.333333\paperwidth,ht=2.25ex,dp=2ex,right]{footercolor}%
				\hspace*{4ex} 
		\end{beamercolorbox}}%
		\vskip0pt%
	}
	
	
	\begin{frame}[noframenumbering]
		%%%%%%%%%%%%%%%%%%%%%%%%%%%%%%%%%%%%%%%%%%%%%%%%%%%%%%%%%%%%
		%%%%%% BREAK BEFORE REFERENCES AND APPENDIX %%%%%%%%%%%%%%%%
		%%%%%%%%%%%%%%%%%%%%%%%%%%%%%%%%%%%%%%%%%%%%%%%%%%%%%%%%%%%%
		\begin{center}
			\Huge{\textcolor{basecolor}{Thank you!}}
		\end{center}
	\end{frame}
	
	\appendix %makes section headers not show up, can make separate section headers for appendix
	\setcounter{section}{0}
	
	
	
	\section*{References}
	\begin{frame}[t,noframenumbering,]{References} %add allowframebreaks if long reference list
		%%%%%%%%%%%%%%%%%%%%%%%%%%%%%%
		%%%%%%%%REFERENCES%%%%%%%%%%%%
		%%%%%%%%%%%%%%%%%%%%%%%%%%%%%%
		%	\tiny
		%	\bibliography{ } %Bib File
		
	\end{frame}
	
	\section*{Tables}
	\begin{frame}[noframenumbering]{Appendix - Tables}
		
	\end{frame}
	
	\section*{Graphs}
	\begin{frame}[noframenumbering]{Appendix - Graphs} \label{appendixgraphs}
		\vfill 
		\begin{center}
			[Graphs]
		\end{center}
		\vfill 
		\hfill 
		\footnotesize{Button back to Introduction \hyperlink{introduction}{\beamerbutton{Back}}}
	\end{frame}
	
\end{document}
